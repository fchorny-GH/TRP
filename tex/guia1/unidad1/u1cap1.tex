 \documentclass[spanish, a4paper, 12pt]{report}
\usepackage[activeacute,spanish,es-lcroman]{babel}
\spanishdecimal{.}
\usepackage{graphicx}
\usepackage[T1]{fontenc}
\usepackage[utf8]{inputenc} %Pone bien los signos de pregunta.
\usepackage{amsmath}
\usepackage{MnSymbol}
\usepackage{wasysym}
\usepackage{amsfonts}
\usepackage{fancyhdr}
\usepackage{comment}
\usepackage{framed}
\usepackage{xcolor}
\usepackage{footnote}
%\usepackage[colorlinks]{hyperref}
\usepackage{longtable}
\usepackage{colortbl}
\usepackage{caption}
\usepackage{subcaption}
\usepackage{multicol}
\setlength\columnsep{8mm}
\usepackage[all]{xy}
\usepackage{pdfpages} %Permite insertar otros pdf o pginas sueltas de un pdf.
\usepackage{pifont}
\usepackage{enumitem}
\usepackage{hyperref}
%--------------------------------
\addto\captionsspanish{
\def\tablename{Tabla}
}
%--------------------------------
\topmargin = -1.25cm
\headheight = 12.766mm
\oddsidemargin = -1.25cm
\evensidemargin = -1.25cm
\textwidth = 17.2cm
\textheight = 23.88cm
\footskip = 14.1mm
\parskip = 1mm
%------Para los pdf digitales------------------------------------------
\usepackage{hyperref} %Genera un pdf con hipervínculos; se pueden programar colores para distintos hipervínculos; acá abajo se dejan todos en negro
\hypersetup{
    colorlinks,
    citecolor=black,
    filecolor=black,
    linkcolor=black,
    urlcolor=blue
}
%--------------------------------
\pagestyle{fancy}
\renewcommand{\headrulewidth}{0pt}
\fancyhf{}
\fancyhead[R]{\resizebox{8cm}{!}{\textsc{Análisis Matemático I (2012), Ingeniería en Electrónica}}}
%---------abreviaturas cómodas-----------------------
\newcommand{\R}{{\mathbb R}}
\newcommand{\C}{{\mathbb C}}
\newcommand{\N}{{\mathbb N}}
\newcommand{\Z}{{\mathbb Z}}
\newcommand{\LL}{{\mathbb L}}
\newcommand{\RenR}[2]{\mathbb{R}^{#1}\longrightarrow \mathbb{R}^{#2}} %Para poner funciones de R^n en R^m
%--------------------------------
%Control de viudas y hurfanas
\clubpenalty=10000
\widowpenalty=10000
%------------------------------------------------
%\renewcommand{\sen}{\sen} 
\newcommand{\cuthere}{%
\noindent
\raisebox{-2.8pt}[0pt][0.75\baselineskip]{\small\ding{34}}
\unskip{\tiny\dotfill}
}
%------------------------------------------------
\renewcommand{\theenumi}{\textbf{Problema \arabic{enumi}.}}
\renewcommand{\labelenumi}{\theenumi}
\renewcommand{\theenumiii}{(\roman{enumiii})}
\renewcommand{\labelenumiii}{\theenumiii}
%-------------------------------------------------
\newcommand{\encabezados}{
\fancypagestyle{plain}{
\fancyhf{}
\fancyhead[R]{\resizebox{8cm}{!}{\textsc{Análisis Matemático I (2012), Ingeniería en Electr\'onica}}}
}
}
\begin{document}
%----------------TEMA 1-----------------------------------
%\encabezados
%\begin{center}
%\textsc{Recuperatorio Parcial 3}\\
%\textsc{Comisión A3}\\
%25/11/2025
%\end{center}
%\vspace{-.5cm}
%\hrulefill
%
%\hspace{-2mm}\fbox{\parbox{0.96\linewidth}{Todas las respuestas deben estar \textbf{justificadas}. Los cálculos deben ir acompañados de explicaciones escritas que aclaren su significado. Un resultado suelto, no acompañado de explicación se considerará como problema no resuelto. Nota de regularización: 3 ítems bien resueltos. Nota de promoción: 2 problemas completos (un $a)$ y un $b)$ bien resueltos).}}

\begin{enumerate}\itemsep=.1cm \itemindent=27mm

\item En un experimento, dos grupos de estudiantes trabajan con un sistema masa-resorte, cada uno. El resorte del grupo 1 responde al modelo \mbox{$f_1(t)=2\sin(0.6 t+0.93)$}. El resorte del grupo 2 responde al modelo \mbox{$f_2(t)=1.5\cos(0.8 t)+2 \sin(0.8 t)$}. 

\includegraphics[width=\linewidth]{ImagenTest.png}

Listas anidadas:

\begin{enumerate}
\item Decidí si el gráfico corresponde a alguno de los dos modelos, explicando cómo te das cuenta y explicá también cuál de los dos resortes fue más apartado de su posición de equilibrio, en el instante en que fue liberado. 
\item Explicá:
	\begin{enumerate}[label=(\roman*)]
	 \item ¿Cuál de los dos resortes fue lanzado con mayor velocidad?
	 \item ¿Cuál de los dos tarda menos tiempo en hacer una oscilación completa?
	 \item ¿Cuál de los dos llega a estirarse más al oscilar?
	 \end{enumerate} 
\end{enumerate}
\end{enumerate}

Algunas ecuaciones:

$$
\int_a^b f(x) dx=F(b)-F(a)
$$

La misma ecuación $\int_a^b f(x) dx=F(b)-F(a)$ en el medio del texto. 
\end{document}
